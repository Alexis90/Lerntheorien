\chapter{Abschließende Bemerkung}
\label{cha:Schluss}
Im Folgenden werden diverse Unterschiede sowie Gemeinsamkeiten hinsichtlich der Rolle \emph{Lernender} sowie \emph{Lehrender} der drei Lerntheorie gegenübergestellt. Dabei wird auch, wie einleitend auf Seite \pageref{cha:Einleitung} erwähnt, eine aus der Autorensicht sinnvolle Zuordnung der didaktischen Modelle zur jeweiligen Theorie vorgenommen. 

Der Kognitivismus wie auch der Konstruktivismus sehen den Lernenden aktiv im Lernprozess involviert. Beide Theorien sprechen von der Verarbeitungsprozessen, welche im Inneren des Menschen beim Lernvorgang vonstatten gehen. Gerade der Kognitivismus legt wie in der Einleitung zu Kapitel \ref{cha:Kognitivismus} näher erläutert, besonderes Augenmerk, und somit stärker als der Konstruktivismus, auf die Informationsverarbeitung während des Lernprozesses. Vergleicht man die beiden Lerntheorien bzgl. der Einflussnahme des Lehrenden auf den Lernerfolg, fällt auf, dass er in der kognitivistischen Lerntheorie größere Einflussnahme auf den Lernerfolg bzw. das Lernergebnis hat als im Konstruktivismus. Hier ist der direkte 1:1 Transfer des Wissens von Lehrende zu Lernende sehr viel schwieriger möglich, da Wissen in dieser Theorie beim Lernenden, wie Kapitel \ref{cha:Konstruktivismus} darlegt, individuell konstruiert wird. Dabei wird im Gegensatz zur kognitivistischen Theorie der Lehrende im Konstruktivismus oft als Berater, Coach oder auch Motivator beschrieben. \cite[S. 30ff.]{Bohm.2006} %FB sehr schön!

Möchte der Lehrende unter Berücksichtigung einer dieser beiden Lerntheorien ein Modell zur didaktischen Aufbereitung des Lernstoffs wählen, so ergibt aus Autorensicht folgende Zuordnung Sinn: Beide Theorien werden idealerweise mit dem didaktischen Modell der Anregung (vgl. \ref{sub:LernenDurchAnregung}) kombiniert. Dies begründet sich mit den in den Theorien verankerten Grundprinzipien der aktiven Auseinandersetzung des Lernenden im Lernprozess mit den Inhalten. Nur in besonderen Ausnahmefällen kann auf das Kopiermodell (vgl. \ref{sub:Kopiermodell}) bspw. bei der Vermittlung von Faktenwissen zurückgegriffen werden. Trotzdem gilt aufgrund neuer Erkenntnisse hinsichtlich verbesserte Lernerfolge durch Nutzung des Anregungsmodells, die Anwendung des Kopiermodells sofern möglich zu vermeiden.

Bei der Recherche für Informationen zum Konnektivismus fiel auf, dass die Einordnung dessen als Lerntheorie von einigen Autoren als kritisch eingestuft wird. Als Beispiel wird hierfür auf die von Kuhlmann und Sauter veröffentlichte Schrift \emph{Innovative Lernsysteme: Kompetenzentwicklung mit Blended Learning und Social Software} (vgl. \cite{Kuhlmann.2008}) sowie auf das von Gaby Filzmoser geschriebende Buch \emph{Bildungshaus 2.0} verwiesen. (vgl. \cite{Filzmoser.2013})

Kuhlmann und Sauter stufen den Konnektivismus nicht als Lerntheorie ein, da er ihrer Meinung nach die traditionellen Theorien erweitert und keine neuen Erkenntnisse hinsichtlich der Informationsverarbeitung durch den Menschen liefert. Der Konnektivismus sei ein pragmatischer Ansatz, welcher die in Abschnitt \ref{Konnektivismus Definition} erwähnten gesellschaftlichen Veränderungen berücksichtigt und als Lernkonzept formuliert .\cite[S. 50]{Kuhlmann.2008} 

Filzmoser bezieht sich auf Plon Verhagen, welcher als Professor für Educational Design an der Universität Twente lehrt und die Einstufung des Konnektivismus als Lerntheorie ebenfalls in Frage stellt. Seiner Meinung nach ist dieser eher Lernphilosphie als -theorie. Er gibt Verhagens Ansicht nach vorwiegend Antworten auf die Frage was gelernt wird und nicht wie gelernt wird.\cite[S. 26]{Filzmoser.2013}

Wie bereits in Abschnitt \ref{Konnektivismus Definition} auf Seite \pageref{RolleLernender} erläutert, ändert sich im Konnektivismus die Ansicht über den Prozess des Lernens. Aus diesem Grund steht nicht die interne Verarbeitung von Informationen im Vordergrund dieser Theorie, sondern hauptsächlich die aktive Interaktion mit Netzwerkknoten. Der Lernende nimmt somit genau wie im Kognitivismus und Konstruktivismus eine aktive, jedoch eine vorwiegend mit seinem Umfeld/Netzwerk dem Austausch gewidmete Rolle ein. Der Lehrende agiert, ähnlich wie im Konstruktivismus, als Coach. 

Die Zuordnung eines der erwähnten didaktischen Aufbereitungsmodelle ist aus Sicht der Autoren für diese Theorie, aufgrund der von verschiedenen Autoren genannten Zweifel zur Einstufung als diese, schwierig. Darüber hinaus wird der Lernprozess vorwiegend als Netzwerkpflege und -ausbauaufgabe gesehen, was die Zuordnung einer der didaktischen Modelle weiter erschwert. Dennoch soll sich nach Meinung der Autoren dieser Arbeit der Lernende idealerweise durch Anregung mit anderen Netzwerkknoten, gerade für die Weiterentwicklung von Wissen, austauschen. Hierbei empfiehlt sich ähnlich dem Anregungsmodell (siehe \ref{sub:LernenDurchAnregung}) vorzugehen.

Eine tabellarische Zusammenfassung mit den in dieser Arbeit abgegebenen Handlungsempfehlung für die jeweilige Lerntheorie kann dem Anhang auf Seite \pageref{fig:Unterschiede und Handlungsempfehlungen für die Lerntheorien} entnommen werden. 


 