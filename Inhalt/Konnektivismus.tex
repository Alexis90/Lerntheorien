\chapter{Konnektivismus}
\label{cha:Konnektivismus}
Der Konnektivismus stellt die dritte und letzte Lerntheorie dar, welche im Rahmen dieser Arbeit vorgestellt wird. Georg Siemens doziert am Red River College in kanadischen Winnipeg und prägte die Lerntheorie 'Connectivism' (zu deutsch: Konnektivismus).\cite[S. 159]{Erpenbeck.2007}

\section{Definition}
Die heutzutage fortschreitende Vernetzung, basierend auf technologischen (Weiter-)Entwicklungen (bspw. das Internet) führt nach Siemens zu einer Veränderung des Lernens. Aus diesem Grund stellt er das Lernen mit Hilfe eines Netzwerkes in den Mittelpunkt. Der Konnektivismus ist eine recht junge Theorie, welche um das Jahr 2006 von Siemens veröffentlicht wurde. Er kritisiert die bis dahin existierenden Lerntheorien Behaviorismus, Kognitivismus (vgl. Kapitel \ref{cha:Kognitivismus}) sowie Konstruktivismus (Kapitel \ref{cha:Konstruktivismus}), da diese seiner Ansicht nach die heutzutage veränderten Rahmenbedingungen der Gesellschaft nicht ausreichend berücksichtigen.\cite[S.47 f.]{Kuhlmann.2008} 

Die 'alten' Lerntheorien suggerieren, dass Erfahrung eine zentrale Rolle im Lernprozess einnimmt. Aufgrund der durch die Digitalisierung geförderten Änderung der Lebens- und Arbeitsweisen, wird es immer schwieriger alle für den Lernprozess erforderlichen Informationsbestandteile sich durch Erfahrung anzueignen.\cite[S. 159ff.]{Erpenbeck.2007} Einige Gründe hierfür sind:
\begin{itemize}
	\item Eine zunehmende Vernetzung führt zur Informationsflut. Daher wird es immer wichtiger zu Wissen wo etwas geschrieben steht und wie relevant es für die Problemlösung ist.
	\item Der Lernende beschäftigt sich aufgrund wirtschaftlicher Gegebenheiten mit verschiedenen (Theorie-)Bereichen.
	\item Individuelle Lernbedürfnisse werden zunehmend relevanter, der Einsatz von E-Learning nimmt zu.
\end{itemize}
Basierend hierauf entwickelt Siemens den Konnektivismus als Lerntheorie, welche als Voraussetzung für die Aneignung von Wissen den Aufbau sowie die Nutzung eines Netzwerkes sieht. Dieses besteht aus Personen, Organisation sowie Datenbanken, welche im weiteren Verlauf als Knoten bezeichnet werden. Somit liegt im Konnektivismus der Fokus der Lernenden auf dem Aufbau und die Pflege dieses Netzwerkes, um stets für ihn relevante, aktuelle sowie für eine Problemlösung adäquate Informationen zugänglich zu machen. Darüberhinaus trägt der Lernende jedoch auch durch sein individuelles Wissen als Quelle zur Aufrechterhaltung dieses Netzwerkes bei und entwickelt dies mit der Interaktion der Netzwerkknoten weiter. Der Lehrende nimmt zunehmend die Rolle des Mentors ein. Er gibt dem Lernenden unter anderem eine Hilfestellung bei der Einordnung des durch das Netzwerk zugänglichen Wissens.\cite[S. 48f.]{Kuhlmann.2008} 

