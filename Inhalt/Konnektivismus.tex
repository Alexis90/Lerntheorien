\chapter{Konnektivismus}
\label{cha:Konnektivismus}
Der Konnektivismus stellt die dritte und letzte Lerntheorie dar, welche im Rahmen dieser Arbeit vorgestellt wird. Georg Siemens doziert am Red River College im kanadischen Winnipeg und prägte die Lerntheorie 'Connectivism' (zu deutsch: Konnektivismus).\cite[S. 159]{Erpenbeck.2007}

\section{Definition}\label{Konnektivismus Definition}
Die heutzutage fortschreitende Vernetzung, basierend auf technologischen (Weiter-) Entwicklungen (bspw. das Internet) führt nach Siemens zu einer Veränderung des Lernens. Aus diesem Grund stellt er das Lernen mit Hilfe eines Netzwerkes in den Mittelpunkt. Der Konnektivismus ist eine recht junge Theorie, welche um das Jahr 2006 von Siemens veröffentlicht wurde. Er kritisiert die bis dahin existierenden Lerntheorien Behaviorismus, Kognitivismus (vgl. Kapitel \ref{cha:Kognitivismus}) sowie Konstruktivismus (Kapitel \ref{cha:Konstruktivismus}), da diese seiner Ansicht nach die heutzutage veränderten Rahmenbedingungen der Gesellschaft nicht ausreichend berücksichtigen.\cite[S.47 f.]{Kuhlmann.2008} 


Die 'alten' Lerntheorien suggerieren, dass Erfahrung eine zentrale Rolle im Lernprozess einnimmt. Aufgrund der durch die Digitalisierung geförderten Änderung der Lebens- und Arbeitsweisen, wird es immer schwieriger alle für den Lernprozess erforderlichen Informationsbestandteile sich durch Erfahrung anzueignen.\cite[S. 159ff.]{Erpenbeck.2007} Einige Gründe hierfür sind:
\begin{itemize} 
	\item Informationsflut durch zunehmende Vernetzung %geändert damit kein Satz sondern Stichpunkt
	\item Beschäftigung mit vielen verschiedenen (Theorie-) Bereichen aufgrund wirtschaftlicher Gegebenheiten %geändert damit kein Satz sondern Stichpunkt
	\item Höhere Relevanz individueller Lernbedürfnisse %geändert damit kein Satz sondern Stichpunkt
\end{itemize}
Basierend hierauf entwickelt Siemens den Konnektivismus als Lerntheorie, welche als Voraussetzung für die Aneignung von Wissen, den Aufbau sowie die Nutzung eines Netzwerkes sieht. Dieses besteht aus Personen, Organisation sowie Datenbanken, welche im weiteren Verlauf als Knoten bezeichnet werden. Somit beschreibt der Konnektivismus hauptsächlich den Lernprozess als den Aufbau und die Pflege dieses Netzwerkes, um stets für den Lernenden aktuelle sowie für eine Problemlösung adäquate Informationen zugänglich zu machen.

\label{RolleLernender}
Dieser muss primär wissen, wo er welche Information findet. Diese muss nicht auswendig gekannt werden. Siemens formuliert darüber hinaus den Grundsatz, dass aktuelles Wissen wichtiger als persönliche Erfahrung ist. Nichtsdestotrotz beschreibt der Konnektivismus den Beitrag des Lernenden durch seine individuelle Erfahrungen, auch erlangt durch die Nutzung des Netzwerkes, sowie ebenfalls durch die Mitteilung von Wertevorstellungen, Emotionen, Denkhaltungen und Interaktion mit den Netzwerkknoten als Lernprozess. Dies dient zur Aufrechterhaltung des in dieser Lerntheorie unabkömmlichen Netzwerkes und bildet die Basis für die Entwicklung neuer Erkenntnisse.

In diesem Modell nimmt der Lehrende hingegen zunehmend die Rolle des Mentors ein. Er gibt dem Lernenden unter anderem eine Hilfestellung bei der Einordnung des durch das Netzwerk zugänglichen Wissens.

Zusammenfassend kann gesagt werden, dass der Konnektivismus informelles sowie formelles Lernen miteinander vermischt. Dies wird erreicht durch die Eingliederung von formellem Lernen (Nutzung von Bildungseinrichtungen für die Wissensaneignung \cite[S. 75]{Hellmer.2007}), sowie durch das Involvieren von informellem Lernen (Lernen im Alltag, ohne die Nutzung von pädagogischen Methoden. Hinzuziehen von Freunden, Familie, Bekannte \cite[S. 76]{Hellmer.2007}) in das oben beschriebene Netzwerk. \cite[S. 47ff.]{Kuhlmann.2008} %Zweite Klammer (Lernen im Alltag ...) besser in einen neuen Satz. Klammerinhalt ist zu lang

\section{Anwendung im E-Learning}
Wie bereits in Abschnitt \ref{Konnektivismus Definition} erwähnt, nimmt die Individualisierung des Lernverhaltens zu. Dies führt zur vermehrten Nutzung von E-Learning und ist einer der Gründe, weshalb sich der Konnektivismus entwickelt hat.\cite[S. 47f.]{Kuhlmann.2008} Im folgenden sollen einige praktische Anwendungsempfehlung ausgesprochen werden, welche Lernen nach dieser Theorie unterstützen. Den Autoren dieser Arbeit erscheint der Einsatz von Kollaborationswerkzeugen in diesem Kontext besonders sinnvoll. %Schwenk zu E-Learning ist etwas abrupt. Einen überleitenden Satz einfügen.

Mit Hilfe eines Forums können Nachrichten mehreren Forum-Teilnehmern zugänglich gemacht, diskutiert, strukturiert sowie archiviert werden. Dies bietet den Vorteil Inhalte zu einem Thema zu sammeln und mit Hilfe der Teilnehmer, wie in Abschnitt \ref{Konnektivismus Definition} erwähnt, weiterzuentwickeln.\cite[S. 67f.]{Drummer.2011}

Eine weitere Möglichkeit der Kollaboration bietet der Einsatz eines Wikis, wie es bereits in Abschnitt \ref{sec:Konstr_Anwendungsfaelle} erläutert wurde. Der Vorteil des Wikis besteht in ihrer flexiblen Gestaltung hinsichtlich Struktur (Einbinden von Mitarbeiterverzeichnissen, Checklisten, Arbeitsanleitungen, Schulungsunterlagen uvm.) und lässt daher viele Anwendungsszenarien zu.\cite[S. 77]{Mertins.2009} Der Einsatz von Foren beschränkt sich jedoch gemäß der obigen Definition hauptsächlich auf Diskussionen.