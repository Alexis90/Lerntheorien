\chapter{Konnektivismus}
\label{cha:Konnektivismus}
Der Konnektivismus stellt die dritte und letzte Lerntheorie dar, welche im Rahmen dieser Arbeit vorgestellt wird. Georg Siemens doziert am Red River College in kanadischen Winnipeg und prägte die Lerntheorie 'Connectivism' (zu deutsch: Konnektivismus).\cite[S. 159]{Erpenbeck.2007}

\section{Definition}
Die heutzutage fortschreitende Vernetzung, basierend auf technologischen (Weiter-)Entwicklungen (bspw. das Internet) führt nach Siemens zu einer Veränderung des Lernens. Aus diesem Grund stellt er das Lernen mit Hilfe eines Netzwerkes in den Mittelpunkt. Der Konnektivismus ist eine recht junge Theorie, welche um das Jahr 2006 von Siemens veröffentlicht wurde. Er kritisiert die bis dahin existierenden Lerntheorien Behaviorismus, Kognitivismus (vgl. Kapitel \ref{cha:Kognitivismus}) sowie Konstruktivismus (Kapitel \ref{cha:Konstruktivismus}), da diese seiner Ansicht nach die heutzutage veränderten Rahmenbedingungen der Gesellschaft nicht ausreichend berücksichtigen.\cite[S.47 f.]{Kuhlmann.2008} 

Die 'alten' Lerntheorien suggerieren, dass Erfahrung eine zentrale Rolle im Lernprozess einnimmt. Aufgrund der durch die Digitalisierung geförderten Änderung der Lebens- und Arbeitsweisen, wird es immer schwieriger alle für den Lernprozess erforderlichen Informationsbestandteile sich durch Erfahrung anzueignen.\cite[S. 159ff.]{Erpenbeck.2007} Einige Gründe hierfür sind:
\begin{itemize}
	\item Eine zunehmende Vernetzung führt zur Informationsflut. 
	\item Der Lernende beschäftigt sich aufgrund wirtschaftlicher Gegebenheiten mit vielen verschiedenen (Theorie-)Bereichen.
	\item Individuelle Lernbedürfnisse werden zunehmend relevanter, der Einsatz von E-Learning nimmt zu.
\end{itemize}
Basierend hierauf entwickelt Siemens den Konnektivismus als Lerntheorie, welche als Voraussetzung für die Aneignung von Wissen den Aufbau sowie die Nutzung eines Netzwerkes sieht. Dieses besteht aus Personen, Organisation sowie Datenbanken, welche im weiteren Verlauf als Knoten bezeichnet werden. Somit beschreibt der Konnektivismus hauptsächlich den Lernprozess als den Aufbau und die Pflege dieses Netzwerkes, um stets für den Lernenden relevante, aktuelle sowie für eine Problemlösung adäquate Informationen zugänglich zu machen. Er muss primär wissen, wo er welche Information findet. Sie muss nicht auswendig gekannt werden. Siemens formuliert darüber hinaus den Grundsatz, dass aktuelles Wissen wichtiger als persönliche Erfahrung ist. Nichtsdestotrotz beschreibt der Konnektivismus den Beitrag des Lernenden durch seine individuelle Erfahrungen, auch erlangt durch die Nutzung des Netzwerkes, sowie ebenfalls durch die Mitteilung von Wertevorstellungen, Emotionen, Denkhaltungen sowie Interaktion mit Netzwerkknoten als Lernprozess. Dies dient zur Aufrechterhaltung des in dieser Lerntheorie unabkömmlichen Netzwerkes und bildet die Basis für die Entwicklung neuer Erkenntnisse. Der Lehrende hingegen nimmt zunehmend die Rolle des Mentors ein. Er gibt dem Lernenden unter anderem eine Hilfestellung bei der Einordnung des durch das Netzwerk zugänglichen Wissens. Zusammenfassend kann gesagt werden, dass der Konnektivismus informelles sowie formelles Lernen miteinander vermischt.\cite[S. 47ff.]{Kuhlmann.2008} Die wird erreicht durch die Eingliederung von formellem Lernen (Nutzung von Bildungseinrichtungen für die Wissensaneignung \cite[S. 75]{Hellmer.2007}), sowie durch das Involvieren von informellem Lernen (Lernen im Alltag, ohne die Nutzung von pädagogischen Methoden. Hinzuziehen von Freunden, Familie, Bekannte \cite[S. 76]{Hellmer.2007}) in das oben beschriebene Netzwerk.  





