\chapter{Customer Relationship Management}
\label{cha:Customer Relationship Management}
Da die im Zuge dieser Projektarbeit behandelte Beratungslösung \ac{EEI} Teil der SAP Implementation eines \ac{CRM} ist, wird im folgenden dieses Konzept behandelt. Die Verbindung zwischen dem SAP \ac{CRM} und der Groupware \ac{MS} Exchange Server (siehe Kapitel \ref{cha:WinServ_ExchServ}) leistet einen Beitrag zu dem Umfang der in SAP \ac{CRM} gespeicherten Informationen. Speziell dieser Mehrwert von Informationen wird auch in diesem Kapitel beleuchtet. (Kapitel \ref{cha:CRM_Info}) Schließlich wird die \ac{CRM}-Lösung der SAP und die Beratungslösung \ac{EEI} beschrieben.

Das \ac{CRM}, bzw. deutsch auch Kundenbindungsmanagement \cite[S.7]{Homburg.2013}, wird geprägt von dem Verständnis um die Kundenbindung eines Unternehmens und seiner Kunden. Kundenbindung hat dabei zum Ziel Intention und tatsächliche Aktion eines Kunden derart zu beeinflussen, dass die Beziehung zwischen diesem und dem Unternehmen verstärkt wird. \cite[S.81-94]{Diller.1996}

\begin{quotation} 
	„Kundenbindungsmanagement ist die systematische Analyse, Planung, Durchführung sowie Kontrolle sämtlicher auf den aktuellen Kundenstamm gerichteten Maßnahmen mit dem Ziel, dass diese Kunden auch in Zukunft die Geschäftsbeziehung aufrechterhalten oder intensiver pflegen.“ 
	
	\cite[S.8]{Homburg.2013}
\end{quotation} 

Allerdings wird \ac{CRM} häufig auch nur als der Ansatz der informationstechnischen Umsetzung von Kundenbindungsmanagement verstanden. \cite[S.3]{Bach.2000} In dieser Projektarbeit wird \ac{CRM} als wissenschaftliches Modell wie in der Definition oben zugrunde gelegt.

Götz und Krafft (\citeyear[S.581]{Gotz.2013}) postulieren, dass die „branchenübergreifend festzustellende Angleichung von Produktvorteilen“ zur Steigerung der Bedeutung von Kundenmanagement-Modellen führt. Daraus lässt sich ableiten, dass das Konzept eines \ac{CRM} heutzutage hohe Aktualität hat, was die steigende Anzahl an Unternehmen, welche ein solches System einführen, erklärte.

\section{Kundenbindung als strategisches Ziel}

Kundenbindung wirkt sich stark auf die Absatzzahlen eines Unternehmens aus und bestimmt so als strategisches Ziel den langfristigen wirtschaftlichen Erfolg mit. \cite[S.17-18]{Homburg.2013}
Dies ist unter anderem an der Bereitschaft von gebundenen Kunden einen höheren Preis zu zahlen zu erkennen. \cite[S. 18]{Homburg.2013} Auch die Kauffrequenz der Kunden steigt und damit die Verkaufsmenge der Produkte. Ein derart treuer Kunde hat für ein Unternehmen einen wesentlich höheren Wert, da dieser in der Kundenbetreuung Kosten spart und eine größere Menge solcher Kunden sich positiv auf das Unternehmensimage auswirkt. \cite[S.18-19]{Homburg.2013}

Darüber hinaus kann die Integration des Kunden in die Prozesse eines Unternehmens  sich fördernd auf die Kundenbindung auswirken. Diese Art von Integration kann bedeuten den Kunden bei Produktentwicklungen einzubinden und auf seine Wünsche einzugehen. \cite[S.157]{Buttgen.2013}
Gute Kundenbindung kann unter der Zuhilfenahme eines \ac{CRM} erreicht werden, in dem die dort verwalteten, aggregierten und analysierten Informationen zur Kommunikation mit und Integration des Kunden genutzt werden. Der Erfolgsfaktor der Kundenbindung \emph{Informationen} wird im nächsten Kapitel genauer beleuchtet.

\section{Erfolgsfaktor Informationen}
\label{cha:CRM_Info}
Laut Alt, Puschmann und Österle (\citeyear[S.2]{Alt.2005}) bilden Informationssysteme und Informationsmanagement einen wichtigen Erfolgsfaktor der Einführung eines \ac{CRM}-Systems in einem Unternehmen. Auch Götz und Krafft (\citeyear[S.591]{Gotz.2013}) sehen im Informationsmanagement einen integralen Erfolgsfaktor für eine \ac{CRM} Implementierung. Hierbei spiele vor allem die Qualität der Informationen, die Integration in bestehende Verwaltungssysteme sowie die Einbindung des Vertriebes eine große Rolle. 

Ein weiterer Erfolgsfaktor ist laut Croteau und Li (\citeyear{Croteau.2003}), dass im Unternehmen Kompetenzen in Sachen Wissensmanagement vorhanden sind. Dies wird darüber definiert Informationen über Kunden, Produkte und Dienstleistungen verwalten und nutzen zu können, um so schnellere und bessere Interaktion mit dem Kunden zu ermöglichen. Unternehmen mit derartiger Kompetenz gelinge es mit Hilfe ihrer vorhandenen IT-Infrastruktur Wissen zu generieren und zu nutzen.

Auch Bach und Österle (\citeyear[S.27-28]{Bach.2000}) messen der Haltung von Information besondere Bedeutung für ein \ac{CRM}-System bei. Sie strukturieren diese in vier Kategorien.

Unter der Kategorie Kundeninformation werden Daten über aktuelle, ehemalige und zukünftige Kunden gepflegt. Hier fließen Daten aus Transaktionen, wie Stammdaten, Aufträge und Verkäufe, mit ein. Außerdem können Informationen über Kunden aus dem direkten Kontakt mit diesen gezogen werden. Dies können persönliche und geschäftliche Ziele, Vorlieben, Interessen und Situation der jeweiligen Kunden bzw. Mitarbeiter des Kunden sein.

Die zweite Kategorie umfasst Produktinformation und hier im speziellen die Information über Produkte und Dienstleistungen, die für Mitarbeiter im Kundenkontakt, wie Kundenberater, wichtig und relevant sind. Dazu gehören sowohl Broschüren, Präsentationen und allgemeine Produktinformationen als auch Absatzzahlen und Verkaufserfahrungen. 

In der Kategorie Kampagneninformation werden Informationen zusammengefasst, welche relevant für Marketingkampagnen sind.  Dabei ist hier die Verbindung zu den anderen Informationskategorien wichtig.

Besonders der Teil eines \ac{CRM} Systems, der sich mit dem Kundenservice befasst, generiert  Serviceinformationen. Dies sind Daten, die häufig aus der Behandlung von Beschwerden oder sonstigen Serviceanfragen bestehen und in Kunden- und Produktinformationen einfließen.
Über Analysen und Interpretationen können aus diesen Daten auch Informationen höherer Aggregationsstufen gewonnen werden.

Zu der \ac{IT}-Infrastruktur gehört auch ein Mail-System, wie es mit \ac{MS} Exchange Server aufgebaut werden kann, so dass sich hier die Nutzung einer Outlook-Integration in \ac{CRM}-Software begründet.  

\section{Geschäftsobjekte eines CRM}

Da innerhalb der hier behandelten Beratungslösung \ac{EEI} auch Informationen mit dem CRM ausgetauscht werden, werden im folgenden einige Informationstypen, genannt Geschäftsobjekte, eines \ac{CRM} exemplarisch dargelegt. 
 
Bei einem Lead handelt es sich um einen möglichen zukünftigen Kunden \cite[S. 41]{BuckEmden.2004} bzw. Kaufinteressenten, zu dem für das Marketing relevante Informationen gesammelt werden, um spätere Akquirierungsmaßnahmen erfolgreich zu gestalten. Aus dem unverbindlichen Lead kann sich im Laufe der Zeit eine Opportunity - das heißt eine \emph{echte} Verkaufschance - ergeben. \cite[S.256-257]{BuckEmden.2004}

Eine Aktivität dokumentiert eine Interaktion mit einem Kunden. Diese werden auch als geplante Folgeaktion für beispielsweise Opportunities angelegt. So kann beispielsweise aus der Opportunity „Kunde XY möchte Produkt AB kaufen“ die Aktivität „Verkaufsgespräch zu Produkt AB bei Kunde XY“ entstehen. Eine Aktivität kann auch die Folgekation zu einem Kundenauftrag oder einem Vertrag sein. \cite[S.237]{Kale.2014}

\section{SAP CRM}

Das SAP \ac{CRM} stellt eine Implementation eines \ac{CRM} durch den Softwarehersteller SAP dar. Diese soll eine Verwaltung der Kundenbeziehungen in den Bereichen Vertrieb, Marketing und Service ermöglichen. \cite[S. 45]{Katta.2009}

\subsection{Architektur}

Basis für SAP \ac{CRM} ist der SAP NetWeaver, welcher Komponenten zur Verfügung stellt SAP Produkte zu betreiben. Wesentlich für den NetWeaver ist der SAP NetWeaver Application Server. \cite[S.241]{Katta.2009} Dieser Application Server unterstützt unter anderem die Protokolle \ac{HTTP} und \ac{HTTPS}, sowie eine Entwicklungsumgebung für Webanwendungen und für Anwendungen die direkt im SAP-Desktop-Client, der SAP GUI, laufen. \cite[S. 247-248]{Katta.2009}
Alle \ac{CRM}-Funktionalitäten können über das SAP \ac{CRM} WebClient abgerufen werden. Die SAP GUI wird für die aktuelle Version des SAP \ac{CRM} lediglich noch zu Konfigurations- und Administrationszwecken genutzt. \cite[S.261]{Katta.2009}

\subsection{SAP CRM Web Client}

Der SAP \ac{CRM} Web Client ist die Benutzeroberfläche des SAP \ac{CRM}, welche ab Version 2007 dort zum Einsatz kommt. Es handelt sich um eine browser-, das heißt auf Webtechnologien, basierende Lösung. \cite[S. 21-22]{Fuchsle.2009}
Der SAP \ac{CRM} Web Client gibt die grundlegende Struktur der Oberfläche vor. Kopfbereich und Navigationsleiste sind vorgegeben und in L-Form angeordnet, weswegen die beiden Komponenten zusammen auch L-Shape genannt werden. \cite[S. 24]{Fuchsle.2009} Die eigentlichen Informationen finden sich im Arbeitsbereich, der den übrigen Platz der Oberfläche einnimmt. \cite[S. 25]{Fuchsle.2009}

\ac{BSP}, eine Web Technologie der SAP, stellt die technische Basis für den Web Client dar. Durch diese Technologie werden Webseiten statischen Layouts mit dynamischem Inhalt erzeugt. Das statische Layout wird durch \ac{HTML}-Quelltext definiert, wohingegen die dynamischen Inhalte via serverseitigem ABAP, einer Programmiersprache der SAP, oder Javascript bereitgestellt werden. \cite[S. 212]{Fuchsle.2009}

\subsection{Enhanced Email Integration}

Die \ac{EEI} ist eine Beratungslösung einer Beratungseinheit innerhalb der SAP Deutschland SE \& Co. KG, welche \ac{MS} Outlook bzw. \ac{MS} Exchange Server in SAP \ac{CRM} integriert, um Nutzerdaten aus diesen Fremdsystemen in SAP \ac{CRM} nutzbar zu machen. \cite[S. 5]{SAPAG.2011}

Im Gegensatz zu der Outlook-Integration, welche in der Standardversion des SAP CRM mitgeliefert wird und nur die Funktionen von Outlook, wie E-Mails senden und empfangen, in das SAP \ac{CRM} überträgt, \cite[S. 67]{Fuchsle.2009} ermöglicht es \ac{EEI} die Daten aus \ac{MS} Outlook in das SAP CRM zur weiteren Verwendung und Generierung von Daten zu nutzen. \cite[S. 6-7]{SAPAG.2011}

\subsubsection{Funktionen}

Im Bereich der Kontaktpflege werden \ac{CRM}-Kontakten ihre E-Mails zugeordnet, um so den Kommunikationsverlauf mit diesen in \ac{CRM} nachvollziehen zu können. Ist zu einer bearbeiteten Email der Kontakt noch nicht vorhanden, wird dieser in \ac{CRM} neu erstellt. Darüber hinaus ermöglicht es \ac{EEI} auch Aktivitäten zu finden, die dem Kontakt der jeweiligen E-Mail zugeordnet sind, und auf Basis einer Email eine Aktivität zu erstellen. \cite[S. 6-7]{SAPAG.2011}

Direkt aus \ac{EEI} heraus ist es außerdem auch möglich verwaltende Tätigkeiten an dem E-Mail-Konto des Nutzers durchzuführen. So kann eine neue E-Mail erstellt und die eigene Ordnerstruktur personalisiert werden. \cite[S. 7]{SAPAG.2011}

\subsubsection{Technische Umsetzung}

Für die Kommunikation zwischen SAP \ac{CRM} bzw. der Beratungslösung \ac{EEI} und dem Exchange Server wird eine Outlook Datei des Typs \ac{DLL} verwendet. \cite[S. 12, 20]{Scharfenberger.2011}
Bei einer \ac{DLL}-Datei handelt es sich um eine Datei, die es Programmen erlaubt Quellcode zu teilen und anderen Programmen Funktionen zur Verfügung zu stellen. \cite{Microsoft.24.09.2011} So erlaubt die hier verwendete Outlook \ac{DLL}-Datei die benötigten Daten zu den E-Mails und Kontakten eines bestimmten Benutzers abzurufen. Außerdem wird die Weiterleitung aus der Anwendung zu der Funktion \emph{E-Mail schreiben} von Outlook unterstützt. \cite[S. 6-7]{SAPAG.2011}
\ac{EEI} wurde als Anwendung für das SAP \ac{CRM} WebClient \ac{UI} gebaut und setzt damit auf die Web-Technologie \ac{BSP} auf. \cite[S. 6]{SAPAG.2011}
