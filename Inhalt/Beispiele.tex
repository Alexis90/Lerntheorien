\chapter{Beispiele}
\label{cha:Beispiele}

\section{Zitieren}

\enquote{Ich bin ein direktes Zitat} nach \citeauthor{Nachname.2013} aus \cite[42]{.2000}.

\enquote{Ich bin ein direktes Zitat} nach \citeauthor{Glover.2006} aus \cite[42]{Glover.2006}.
\section{Verweise}

\autoref{cha:Beispiele} zeigt ab Seite~\pageref{cha:Beispiele} Beispiele zur Syntax von \LaTeX{}.

Titel zitieren: \citetitle{Nachname.2013}

\section{Aufzählungen}
\label{sec:Aufzaehlungen}

\subsection{Itemize}

\begin{itemize}
	\item
		Ich bin das erste Item
	\item
		Ich bin das zweite Item.
\end{itemize}description

\subsection{Enumerate}

\begin{enumerate}
	\item
		Ich bin das erste Item
	\item
		Ich bin das zweite Item.
\end{enumerate}

\subsection{Description}

\begin{description}
	\item[Item 1]
		Ich bin das erste Item
	\item[Item 2]
		Ich bin das zweite Item.
\end{description}

\section{Abbildungen}

In \autoref{fig:LogoDHBW Karlsruhe} auf Seite~\pageref{fig:LogoDHBW Karlsruhe} ist das Logo der DHBW Karlsruhe zu sehen.

\begin{figure}[!htbp]
	\centering
	\includegraphics[scale=0.75]{Abbildungen/Logo_DHBW_Karlsruhe.png}
	\caption{Logo der DHBW Karlsruhe}
	\label{fig:LogoDHBW Karlsruhe}
\end{figure}

\section{Tabellen}

Tabelle~\ref{tbl:TestTabelle} auf Seite~\pageref{tbl:TestTabelle} zeigt eine Test-Tabelle.

\begin{table}[!htbp]
	\centering
	\begin{tabular}{ll}
		\textbf{Test}	& \textbf{Test}\\
		\hline
		Test 	&	Test\\
		Test 	&	Test\\
		\hline
	\end{tabular}
	\caption{Test-Tabelle}
	\label{tbl:TestTabelle}
\end{table}

\begin{table}[!htbp]
	\centering
	\begin{tabular}{ll}
		\textbf{Test}	& \textbf{Test}\\
		\hline
		Test 	&	Test\\
		Test 	&	Test\\
		\hline
	\end{tabular}
	\caption{Test-Tabelle 2}
	\label{tbl:TestTabelle2}
\end{table}

\section{Abkürzungen}

\ac{T} ist eine Abkürzung.
\ac{T} wird beim ersten Vorkommen zur Erklärung ausgeschrieben.

Und noch eine Abkürzung: \ac{TT}.