% Hinweis: Optionen der Dokumentenklasse werden an alle folgenden \usepackage{package} Befehle weitergegeben
\documentclass[
	pdftex,
	fontsize=12pt,
	paper=a4,
	halfparskip,
	twoside=false,
	numbers=noenddot,	% Kein Punkt am Ende einer Überschrift
	%draft=true,			% Deckt Schwächen auf: overfull und full boxes werden markiert; Bilder werden nicht geladen
	bibliography=totoc,	% Literaturverzeichnis ins Inhaltsverzeichnis aufnehmen
	listof=totoc,		% Tabellen- und Abbildungsverzeichnis ins Inhaltsverzeichnis aufnehmen
	titlepage=true,		% Separate Titelseite; Gestaltung mit Hilfe der Titlepage-Umgebung
	headsepline=true,	% Kopflinie aktivieren
	footsepline=true	,	% Fußlinie aktivieren
	abstracton			% Abstract aktivieren
]{scrreprt}

% Zeichenkodierung Input ist UTF-8: Umlaute können direkt eingegeben werden
\usepackage[utf8]{inputenc}

% Zeichenkodierung Ausgabe ist T1-Kodierung: Wichtig für die Ausgabe von Umlauten
\usepackage[T1]{fontenc}

% Schrift festlegen
\usepackage{lmodern}

% Sprachauswahl für Lokalisierungen und Silbentrennung
\usepackage[ngerman]{babel}

% Zitate: Anführungszeichen automatisch anhand der Sprache wählen
\usepackage[babel=true]{csquotes}

% Source-Code-Listings
\usepackage{listings}

% BibTeX-Symbol
\usepackage{texnames}

\usepackage{xcolor}

% Symbole, z.B. Haken
\usepackage{pifont}

% Zeilen in Tabellen zusammenfassen
\usepackage{multirow}

% Silbentrennung kann bei bestimmten Wörten mit Hilfe von diesem Paket deaktiviert werden 
\usepackage{hyphenat}

% Abkürzungsverzeichnis
%\usepackage[printonlyused, withpage]{acronym}
% Abstand mit Punkten füllen
%\renewcommand*\aclabelfont[1]{\textbf{\normalsize{#1}}\hfill}

% Tiefe des Inhaltsverzeichnisses
\setcounter{tocdepth}{2}

% Punkte im Inhaltsverzeichnis
\usepackage{tocstyle}
\usetocstyle{allwithdot}

% Zum Einbinden von PDF-Dateien.
\usepackage{pdfpages}

% Paket zum Anpassen von Kopf- und Fußzeilen
\usepackage[plainfootsepline, plainheadsepline, headsepline, footsepline, automark]{scrpage2}
% Liniendicke
\setheadsepline{0.1pt}
\setfootsepline{0.1pt}

% Kopf- und Fusszeile löschen
\clearscrheadfoot
% Kopf- und Fusszeile aktivieren
\pagestyle{scrheadings}

% Kopf links
\ihead[\titel]{\titel}

% Fuss links
\ifoot[\verfasser]{\verfasser}
% Fuss rechts
\ofoot[\pagemark]{\pagemark}

% Grafiken einbinden
\usepackage{graphicx}
% Pfad zu den Grafiken
\graphicspath{{Abbildungen/}}

% Seitenränder setzen
\usepackage[left=3cm, right=2cm, top=2.5cm, bottom=3cm]{geometry}

% Zeilenabstand auf 1.5 setzen
\usepackage{setspace}
\onehalfspacing

% Literaturverzeichnis
%\usepackage[backend=bibtex,style=authoryear]{biblatex}
%\usepackage{bibgerm}
%\usepackage{natbib}
\usepackage[backend=bibtex,style=alphabetic]{biblatex}
%\bibliographystyle{alphadin}
\bibliography{Literatur/Literatur.bib}

% Glossar
\usepackage[toc,nonumberlist]{glossaries}

% Titel als Referenzierung verwenden
\usepackage{titleref}

% Währungen
\usepackage{textcomp}

% Fussnoten fortlaufend nummerieren.
\usepackage{chngcntr}
\counterwithout{footnote}{chapter}

% Persönliche Daten
\newcommand{\titel}{Lerntheorien: Kognitivismus, Konstruktivismus und Konnektivismus}
\newcommand{\untertitel}{}
\newcommand{\art}{Seminararbeit}
\newcommand{\verfasser}{Alexander Hauck und Fabian Berns}
\newcommand{\kurs}{WWI13B5}
\newcommand{\ausbildungsbetrieb}{SAP SE}%, Dietmar-Hopp-Allee 16, 69190 Walldorf}
\newcommand{\abgabedatum}{__________________}

\newcommand{\martrikelnr}{}
\newcommand{\betreuer}{Frau Judith Hüther}
\newcommand{\gutachter}{}
\newcommand{\zeitraum}{November 2015 - Januar 2016}


% Links- und PDF-Einstellungen
\usepackage{hyperref}
\hypersetup{
	pdfauthor = {\verfasser},
	pdftitle = {\titel},
	pdfsubject = {\art},
	pdfkeywords = {},
	pdfstartview = {Fit},
	colorlinks = {false},
	breaklinks = {true},
	bookmarksopen = {true}
}

% Verhinderung von Schusterjunge und Hurenkind
\clubpenalty = 10000
\widowpenalty = 10000
\displaywidowpenalty = 10000

% Seitenzäler für große, römische Zahlen
\newcounter{RomanPagenumber}

% Abkürzungen
\newcommand{\dash}{d.\,h.}
\newcommand{\zB}{z.\,B.}

%Für Javascript Code
\usepackage{color}
\definecolor{lightgray}{rgb}{.9,.9,.9}
\definecolor{darkgray}{rgb}{.4,.4,.4}
\definecolor{purple}{rgb}{0.65, 0.12, 0.82}

\lstdefinelanguage{JavaScript}{
	keywords={typeof, new, true, false, catch, function, return, null, catch, switch, var, if, in, while, do, else, case, break},
	keywordstyle=\color{blue}\bfseries,
	ndkeywords={class, export, boolean, throw, implements, import, this},
	ndkeywordstyle=\color{darkgray}\bfseries,
	identifierstyle=\color{black},
	sensitive=false,
	comment=[l]{//},
	morecomment=[s]{/*}{*/},
	commentstyle=\color{purple}\ttfamily,
	morestring=[b]',
	morestring=[b]",
}